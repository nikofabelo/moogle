\section{Estructura y Algoritmo}

\begin{frame}
    \begin{columns}[t]
        \begin{column}{.5\textwidth}
          \tableofcontents[sections={1-2},currentsection]
        \end{column}
        \begin{column}{.5\textwidth}
          \tableofcontents[sections={3-4},currentsection]
        \end{column}
    \end{columns}
\end{frame}

\subsection{Estructura y Algoritmo}
\begin{frame}[fragile]
Moogle! surge de dos componentes principales: \texttt{MoogleEngine} y \texttt{MoogleServer}.
\newline El componente \texttt{MoogleServer} es el encargado de servir las páginas web, su funcionamiento no es sofisticado
y la parte encargada de calcular los resultados de las búsquedas se encuentra totalmente implementada en \texttt{MoogleEngine}.
\end{frame}

\begin{frame}[fragile]
El programa funciona en su conjunto de la siguiente manera.
Dentro de la clase \texttt{Moogle} se cargan los documentos la primera vez que se hace una búsqueda,
esto se consigue mediante el uso de un valor booleano que permite que el 
procesamiento de los documentos ocurra solo una vez. De esta 
manera no hay necesidad de tener que procesar los documentos cada 
vez que se hace una nueva búsqueda.
\end{frame}

\begin{frame}[fragile]
Primero que todo, se crea un objeto \texttt{Corpus} que carga todos los 
documentos de la carpeta Content. Esto lo hace creando un array de objetos \texttt{Document}. En
el momento de inicialización de los objetos \texttt{Document} estos 
dividen el texto del documento al que hacen referencia en 
palabras, lo llevan a minúsculas y eliminan carácteres especiales para mejorar las búsquedas.
También se calcula el TF o frequencia de término de cada 
palabra, que se va añadiendo a un diccionario. Cuando una 
palabra aparece por primera vez en cada documento distinto se 
aumenta también su frecuencia en otro diccionario DTF, que 
indica la cantidad de documentos en los que aparece esa palabra.
Luego de que se lean todos los documentos se procede a calcular 
el IDF de cada palabra. Para con todos estos datos crear un objeto
\texttt{Matrix}, que no es más que el recipiente de un arreglo de objetos 
\texttt{Vector} que representan los vectores de TF-IDF de los 
documentos, estos vectores tienen el tamaño del banco de 
palabras.
\end{frame}

\begin{frame}[fragile]
Una vez que se ha creado la matriz de vectores que representan 
los documentos se procede a crear un \texttt{Vector} que represente al 
\texttt{Query}, de dimensión igual a la dimensión de los vectores de los documentos.
En este sentido, el objeto \texttt{Matrix} y el objeto \texttt{Query} solo son
formas distintas de acceder a los vectores de TF-IDF que 
representan la información.
\pause \newline
Cuando se tienen todos los vectores se calcula la similitud 
cosénica entre el vector de búsqueda y los vectores de los documentos, se ordenan los resultados de la búsqueda según este valor dando prioridad a
los valores más altos y seguidamente se envía un resultado de búsqueda mediante una clase \texttt{SearchResult} (que contiene instancias de
otra clase \texttt{SearchItem}) a la página web, que lo maneja apropiadamente.
\end{frame}





% \begin{lstlisting}[basicstyle=\tiny]
% \begin{center}
%   \text{\Large My project title}\\
%   \text{\small John Doe}\\
%   \vspace{0.5cm}
%   \text{\tiny Jul 14th, 2023}\\
% \end{center}
% \end{lstlisting}
