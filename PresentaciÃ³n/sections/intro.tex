\section{\textquestiondown Qu\'e es ``Moogle!''?}

\begin{frame}
    Moogle! consiste en un motor de búsqueda y servidor web que sirve páginas de internet
utilizando las tecnologías .NET y Blazor. Permite la búsqueda de un texto en un conjunto de
documentos en formato \texttt{.txt} (text/plain), dentro de una carpeta Content. También permite
acceder a estos documentos al hacer click encima de sus nombres de archivo que aparecen en color
verde en la página de búsqueda. Siempre que los resultados de una búsqueda sean menores que 5, Moogle!
recomendará una sugerencia de búsqueda con mayor probabilidad de aparecer en el cuerpo de documentos.
    Para el funcionamiento de Moogle! se utiliza un \texttt{Modelo de Espacio Vectorial}\footnote{
\url{https://es.wikipedia.org/wiki/Modelo_de_espacio_vectorial}}, y para medir la importancia de cada término determinado
en el cuerpo de documentos se utiliza un valor de \texttt{TF-IDF}\footnote{\url{https://es.wikipedia.org/wiki/Tf-idf}}.
\end{frame}


\begin{frame}
    \begin{columns}[t]
        \begin{column}{.5\textwidth}
          \tableofcontents[sections={1-2},currentsection]
        \end{column}
        \begin{column}{.5\textwidth}
          \tableofcontents[sections={3-4},currentsection]
        \end{column}
    \end{columns}
\end{frame}


\subsection{Ventajas y desventajas}
\begin{frame}{Ventajas y desventajas}

Ventajas:
\begin{itemize}
  \item Representación eficiente de documentos
  \item Búsqueda basada en relevancia
  \item Flexibilidad en la consulta
  \item Escalabilidad
  \item Personalización y recomendaciones
\end{itemize}

\pause

Desventajas:
\begin{itemize}
  \item Limitaciones en la representación semántica
  \item Sensibilidad a la calidad de los datos de texto
\end{itemize}

\end{frame}

\subsection{Ejecución}

\begin{frame}{\textquestiondown Cómo ejecutar ``Moogle!''?}

\begin{block}{Linux}
  En la raíz del proyecto ejecutar:\\[2mm]
  \textbf{Ubuntu}: \texttt{make dev}\\
\end{block}

\begin{block}{Cualquier otro SO}
  En la raíz del proyecto ejecutar:\\[2mm]
  \texttt{dotnet watch run --project MoogleServer}\\
\end{block}

\end{frame}
