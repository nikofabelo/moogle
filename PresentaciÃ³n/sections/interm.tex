\section{MoogleEngine}
\begin{frame}
    \begin{columns}[t]
        \begin{column}{.5\textwidth}
          \tableofcontents[sections={1-2},currentsection]
        \end{column}
        \begin{column}{.5\textwidth}
          \tableofcontents[sections={3-4},currentsection]
        \end{column}
    \end{columns}
\end{frame}

\begin{frame}[fragile]
Este se compone de 6 clases importantes: \texttt{Corpus}, \texttt{Document}, \texttt{Matrix},
\texttt{Vector}, \texttt{Query} y la clase \texttt{Moogle}, que contiene un método \texttt{Query}
(el cual se encarga de procesar una búsqueda y de devolver sus resultados a la página web).
\end{frame}

\subsection{Moogle}
\begin{frame}[fragile]{Moogle}
    Constituye la clase principal de Moogle! pues contiene el método \texttt{Query!} que es el encargado de orquestar
las funciones del motor de búsqueda. Primeramente se crea una instancia de \texttt{Corpus} que leerá los archivos de la carpeta
Content y luego se crea una instancia de \texttt{Matrix} a partir de estos archivos. Como que solo se quiere realizar este proceso
una vez entonces se hace uso de un flag que indica si el motor de búsqueda ha sido inicializado, al llamar al método \texttt{Query}
este revisa si el motor ya se ha iniciado y en dicho caso no lo vuelve a iniciar. Una vez se tiene la matriz del corpus se toma una
instancia del vector del query para con esta y los vectores documento calcular la similitud cosénica cuya fórmula es (1):
\begin{equation}
    cos = \frac{v \cdot w}{||v|| \cdot ||w||}
\end{equation}\newline
\end{frame}
\begin{frame}[fragile]{Moogle}
Los valores de similitud cosénica mayores que 0, osea, que tienen al menos una relevancia mínima para la búsqueda, son guardados en un
diccionario que toma como llaves los objetos \texttt{Document}, el cual posteriormente se ordena descendientemente y es usado para crear
instancias de \texttt{SearchItem} que se utilizan para inicializar un objeto \texttt{SearchResult}. Si la cantidad de resultados es menor
que 5 entonces se calcula una sugerencia de búsqueda, para esto su utiliza el cálculo de distancia de Levenshtein. Se hace uso de una lista e
iterando sobre las palabras del query se busca dentro del banco de palabras las palabras distintas y con mayor similitud a esta, en caso de encontrarse
una de estas palabras se reemplaza en el query, el resultado final es un query con mayor probabilidad de arrojar resultados en la búsqueda. Para determinar
cuan similiar es una palabra con otra usando Levenshtein se toman en cuenta la cantidad de inserciones, eliminaciones o substituciones de carácteres que
se deben llevar a cabo entre dos palabras para que sean iguales, a menor valor de estas entonces más similares son las palabras entre sí. Para calcular
la distancia de Levenshtein se crea una matriz cuyo tamaño es $(m + 1)\times (n + 1)$ donde $m$ y $n$ son el número de carácteres de cada cadena. Después
de creada esta matriz se rellenan su primera fila y columna con los casos base. En caso de que los carácteres de las palabras sean iguales para sus determinados índices se mantiene el valor de distancia que los separa y en caso contrario se aumenta esta. Finalmente tanto el objeto \texttt{SearchResult}
como la sugerencia se envían a la página que ha de mostrarlos.
\end{frame}






\subsection{Corpus}
\begin{frame}[fragile]{Corpus}
    Al conjunto de todos los documentos que aparezcan en la carpeta Content haremos referencia como \texttt{Corpus}.
Este como clase contiene dos diccionarios: DTF e IDF, los cuales hacen referencia a la cantidad de veces que una palabra
aparece en el corpus y al valor de IDF para un término en específico, respectivamente. Primero que todo se inicializan
los documentos de la carpeta Content. Para ello se lee la lista de archivos de la carpeta Content, se crea un array de
\texttt{Document} y se inicializan los documentos con las rutas de la lista de archivos.
    Acto seguido se calcula el IDF (Inverse Document Frequency) para cada término. Inverse Document Frequency es un valor
que se utiliza para contrarrestar en el TF-IDF términos que puedan tener valores de TF muy altos como artículos, preposiciones
y conjunciones. La fórmula mediante la cual este se calcula es la siguiente (2):
\begin{equation}
    IDF_t = \log\left(\frac{\text{Cantidad de Documentos}}{DTF_t}\right)
\end{equation}
\end{frame}

\subsection{Document}
\begin{frame}[fragile]{Document}
    Esta clase proporciona acceso al TF de los términos de los documentos, además de ocuparse de la lectura y
preprocesamiento del texto de los archivos. Cada documento está asociado a un corpus de documentos, de esta manera puede
registrar valores de su diccionario DTF. En comienzo se lee el archivo y se extraen sus palabras y frecuencias a partir
de una ruta de archivo. Si la ruta al archivo no comienza con el prefijo especial que determina los objetos \texttt{Query} entonces
se intenta procesar el texto del archivo indicado por la ruta, en caso de no poderse se lanza una excepción. Antes que todo se leen todas
las líneas de texto del archivo usando la codificación UTF-8, se eliminan todos los carácteres no alfanuméricos y se lleva todo a minúsculas,
separando por espacios el texto y eliminando los términos que consistan solo de carácteres blancos, obteniéndose un array de \texttt{string},
llamado \texttt{words}. Después de esto se obtiene un snippet de un mínimo de 100 palabras del documento (siempre que sea posible), lo cual se
logra leyendo lineas del archivo al que hace referencia la ruta de archivo hasta que la cantidad de palabras del snippet deje de ser menor de 100.
    Por otra parte si la ruta de archivo proporcionada comienza con el prefijo inicial que indica que es el texto de un query, entonces
se elimina el prefijo especial y se hace el mismo preprocesamiento sobre el texto del query, eliminando carácteres y llevando a minúsculas.
\end{frame}

\subsection{Matrix}
\begin{frame}[fragile]{Matrix}
    Esta clase actúa como un recipiente, ya que ella brinda una forma de acceder e indexar los
vectores de todos los documentos del corpus. Para esto simplemente crea un array de objetos \texttt{Vector}
que se relacionan a los vectores de TF-IDF de cada documento.
\end{frame}

\subsection{Vector}
\begin{frame}[fragile]{Vector}
    La clase \texttt{Vector} define un vector de TF-IDF para cada documento. Este vector tiene el tamaño del
banco de palabras y sus componentes consisten en el TF de cada término en el documento, multiplicado por su IDF.
Al momento de crearse también se computa su norma, la cual es necesaria en el cálculo de la similitud
cosénica y su fórmula es (3):
\begin{equation}
    ||v|| = \sqrt{\sum_{i=1}^{n} x_i^2}
\end{equation}
\end{frame}


\subsection{Query}
\begin{frame}[fragile]{Query}
    Con el objetivo de simplificar el programa esta clase consiste fundamentalmente
en brindar una capa de abstracción para tratar el texto de una búsqueda como si fuese
un objeto \texttt{Document}. Esto se logra inicializando un objeto \texttt{Document} con
una ruta de archivo compuesta de un prefijo especial seguido del texto de la búsqueda y que es
reconocida por la clase \texttt{Document} durante su inicialización.
\end{frame}
